% Preface required in the knitr RnW file
\documentclass{article}\usepackage[]{graphicx}\usepackage[]{color}
%% maxwidth is the original width if it is less than linewidth
%% otherwise use linewidth (to make sure the graphics do not exceed the margin)
\makeatletter
\def\maxwidth{ %
  \ifdim\Gin@nat@width>\linewidth
    \linewidth
  \else
    \Gin@nat@width
  \fi
}
\makeatother

\definecolor{fgcolor}{rgb}{0.345, 0.345, 0.345}
\newcommand{\hlnum}[1]{\textcolor[rgb]{0.686,0.059,0.569}{#1}}%
\newcommand{\hlstr}[1]{\textcolor[rgb]{0.192,0.494,0.8}{#1}}%
\newcommand{\hlcom}[1]{\textcolor[rgb]{0.678,0.584,0.686}{\textit{#1}}}%
\newcommand{\hlopt}[1]{\textcolor[rgb]{0,0,0}{#1}}%
\newcommand{\hlstd}[1]{\textcolor[rgb]{0.345,0.345,0.345}{#1}}%
\newcommand{\hlkwa}[1]{\textcolor[rgb]{0.161,0.373,0.58}{\textbf{#1}}}%
\newcommand{\hlkwb}[1]{\textcolor[rgb]{0.69,0.353,0.396}{#1}}%
\newcommand{\hlkwc}[1]{\textcolor[rgb]{0.333,0.667,0.333}{#1}}%
\newcommand{\hlkwd}[1]{\textcolor[rgb]{0.737,0.353,0.396}{\textbf{#1}}}%
\let\hlipl\hlkwb

\usepackage{framed}
\makeatletter
\newenvironment{kframe}{%
 \def\at@end@of@kframe{}%
 \ifinner\ifhmode%
  \def\at@end@of@kframe{\end{minipage}}%
  \begin{minipage}{\columnwidth}%
 \fi\fi%
 \def\FrameCommand##1{\hskip\@totalleftmargin \hskip-\fboxsep
 \colorbox{shadecolor}{##1}\hskip-\fboxsep
     % There is no \\@totalrightmargin, so:
     \hskip-\linewidth \hskip-\@totalleftmargin \hskip\columnwidth}%
 \MakeFramed {\advance\hsize-\width
   \@totalleftmargin\z@ \linewidth\hsize
   \@setminipage}}%
 {\par\unskip\endMakeFramed%
 \at@end@of@kframe}
\makeatother

\definecolor{shadecolor}{rgb}{.97, .97, .97}
\definecolor{messagecolor}{rgb}{0, 0, 0}
\definecolor{warningcolor}{rgb}{1, 0, 1}
\definecolor{errorcolor}{rgb}{1, 0, 0}
\newenvironment{knitrout}{}{} % an empty environment to be redefined in TeX

\usepackage{alltt}

\usepackage{rotating}
\usepackage{graphics}
\usepackage{latexsym}
\usepackage{color}
\usepackage{listings} % allows for importing code scripts into the tex file
\usepackage{amsmath}

% Approximately 1 inch borders all around
\setlength\topmargin{-.56in}
\setlength\evensidemargin{0in}
\setlength\oddsidemargin{0in}
\setlength\textwidth{6.49in}
\setlength\textheight{8.6in}

% change name of Abstract
\renewcommand{\abstractname}{Executive Summary}
% allow multiple sections in appendix
\usepackage[title]{appendix}

\title{The Effect of New Housing Projects on Expenditures in Two New York Municipalities} 
\author{Andrew Bates}
\date{October 11, 2018}
\IfFileExists{upquote.sty}{\usepackage{upquote}}{}
\begin{document} 
\maketitle

\begin{abstract}
In this paper we estimate future expenditures for two municipalites in New York based on various projected demographic and income-related factors. These estimates are obtained from a linear model chosen via a stepwise regression procedure with AIC as the model selection criteria. The variables in the model are wealth per person, population, percent intergovernmental funding, and growth rate. The estimated expenditures for Warwick are 1, 2 for the years 2005 and 2025. The estimated expenditures for Monroe are 1, 2 for the years 2005 and 2025.
\end{abstract}


\section{Introduction} \label{intro}

Two New York towns, Warwick and Monroe, would like to estimate future expenditures triggered by new housing construction proposals. They are primarily interested in determining whether they need to increase funds to compensate for increased expenditures related to the housing projects. To construct these estimates, Warwick and Monroe obtained data on expenditures along with various demographic and income-related variables from several New York municiplaties. 

\section{Methods} \label{methods}

The data used in this study consisted of 916 observations of seven variables. Each observation corresponds to a New York municipality for which each of the variables were collected. The response variable is expenditure per person. The covariates are as follows: wealth per person, population, revenue from state and federal grants, population density, mean income per person, and growth rate.

For reasons of simplicity and interpretability, a linear regression model was chosen to estimate future expenditures. The correlation between population and population density was high ($0.67$) so to prevent this from leading to multicolinearity issues, only population was considered in the analysis. The variables in this data set (including the response) were heavily skewed, so log transformations were applied to each. (A log transformation was chosen over a more complex transformation to aid in interpretation). To ensure a linear relationship between the predictors and the response, the data was subsetted to include only those observations for which the population was larger than 4,000 (log population above $8.3$). Since the projected covariates for Warwick and Monroe fall within the range of the covariates in the subsetted data, this should not lead to any problems. In addition, this method was favored over a more complicated method like including a quadratic term on log population which would be harder to interpret. The regression model was constructed via stepwise regression using AIC as the selection criteria.



\section{Analysis} \label{analysis}




\subsection{Exploratory Data Analysis} \label{eda}

In this data set there were two measures of the size of a municipality: population and population density. Unsurprisingly, the correlation between these two measures was relatively high ($0.67$). In the interest of parsimony and to mitigate possible colinearity issues, we decided to consider only one of these variables to construct our model. Population density had a moderate correlation with mean income per person ($0.49$) whereas population had a comparably low correlation with income ($0.29$). To hedge against problems with colinearity, we decided to consider population for the model building process.

Upon an initial examination of the data, it was evident that transformations would be necessary. Each of the covariates, and the response, were skewed (see Appendix \ref{appendix_plots}) and some were heavily skewed. As this might pose problems with linearity, we performed transformations on the variables. To remain in line with our goal of having a comprehensible model, we favored log transformations over a more complex procedure. To that end, we performed log transformations on all variables in the data set.  However, a straightforward application of the logarithm was not possible for one covariate. Growth rate contains some negative values as well as some zero values. To remedy this, we used the following pseudo-log transformation:
\[
\text{p-}\log(\text{growth rate}) = % plog for pseudo-log
  \begin{cases}
    \log(\text{growth rate} + 0.15) &\text{if growth rate} > 0 \\
    -\log(-\text{growth rate} + 0.15) &\text{if growth rate} \le 0.
  \end{cases}
\]
Note that since this is a one-to-one transformation ...

To ensure the linearity assumption of our model was satisfied, we examined the relationship between log expenditure and the log of each of the covariates. Except for log population, all other covariates had an approximately liner relationship with log expenditure. Figure () is a plot of log expenditure vs. log population.


\subsection{Modeling and Diagnostics} \label{modeling}

\section{Conclusion} \label{conclusion}



\newpage
\begin{appendices}

\section{Supplementary Plots} \label{appendix_plots}

\end{appendices}


\end{document}
