% Preface required in the knitr RnW file
\documentclass{article}\usepackage[]{graphicx}\usepackage[]{color}
%% maxwidth is the original width if it is less than linewidth
%% otherwise use linewidth (to make sure the graphics do not exceed the margin)
\makeatletter
\def\maxwidth{ %
  \ifdim\Gin@nat@width>\linewidth
    \linewidth
  \else
    \Gin@nat@width
  \fi
}
\makeatother

\definecolor{fgcolor}{rgb}{0.345, 0.345, 0.345}
\newcommand{\hlnum}[1]{\textcolor[rgb]{0.686,0.059,0.569}{#1}}%
\newcommand{\hlstr}[1]{\textcolor[rgb]{0.192,0.494,0.8}{#1}}%
\newcommand{\hlcom}[1]{\textcolor[rgb]{0.678,0.584,0.686}{\textit{#1}}}%
\newcommand{\hlopt}[1]{\textcolor[rgb]{0,0,0}{#1}}%
\newcommand{\hlstd}[1]{\textcolor[rgb]{0.345,0.345,0.345}{#1}}%
\newcommand{\hlkwa}[1]{\textcolor[rgb]{0.161,0.373,0.58}{\textbf{#1}}}%
\newcommand{\hlkwb}[1]{\textcolor[rgb]{0.69,0.353,0.396}{#1}}%
\newcommand{\hlkwc}[1]{\textcolor[rgb]{0.333,0.667,0.333}{#1}}%
\newcommand{\hlkwd}[1]{\textcolor[rgb]{0.737,0.353,0.396}{\textbf{#1}}}%
\let\hlipl\hlkwb

\usepackage{framed}
\makeatletter
\newenvironment{kframe}{%
 \def\at@end@of@kframe{}%
 \ifinner\ifhmode%
  \def\at@end@of@kframe{\end{minipage}}%
  \begin{minipage}{\columnwidth}%
 \fi\fi%
 \def\FrameCommand##1{\hskip\@totalleftmargin \hskip-\fboxsep
 \colorbox{shadecolor}{##1}\hskip-\fboxsep
     % There is no \\@totalrightmargin, so:
     \hskip-\linewidth \hskip-\@totalleftmargin \hskip\columnwidth}%
 \MakeFramed {\advance\hsize-\width
   \@totalleftmargin\z@ \linewidth\hsize
   \@setminipage}}%
 {\par\unskip\endMakeFramed%
 \at@end@of@kframe}
\makeatother

\definecolor{shadecolor}{rgb}{.97, .97, .97}
\definecolor{messagecolor}{rgb}{0, 0, 0}
\definecolor{warningcolor}{rgb}{1, 0, 1}
\definecolor{errorcolor}{rgb}{1, 0, 0}
\newenvironment{knitrout}{}{} % an empty environment to be redefined in TeX

\usepackage{alltt}

\usepackage{rotating}
\usepackage{graphics}
\usepackage{latexsym}
\usepackage{color}
\usepackage{listings} % allows for importing code scripts into the tex file
\usepackage{amsmath}

% Approximately 1 inch borders all around
\setlength\topmargin{-.56in}
\setlength\evensidemargin{0in}
\setlength\oddsidemargin{0in}
\setlength\textwidth{6.49in}
\setlength\textheight{8.6in}


\title{Stat 696, Example Application of \texttt{knitr}} 
\author{YOUR NAME}
\date{\today}
\IfFileExists{upquote.sty}{\usepackage{upquote}}{}
\begin{document} 
\maketitle


The \textit{insert reason here. e.g. skewness} of growth rate suggested a log transformation. However, growth rate contains some negative values as well as some zero values. For this reason, we used the following modified log transformation:

% for some reason this is throwing an error
% undefined control sequence, misplaced alignment tab character &
% seems like maybe it's not recognizing cases environment
% needed to add \usepackage{amsmath}
% I guess I'm just too used to this already being there
% or not needing to do it b/c R Markdown makes this available!
\[
p\log(\text{growth rate}) = % plog for pseudo-log
  \begin{cases}
    \log(growth rate + 0.15) &\text{if growth rate} > 0 \\
    -\log(-growth rate + 0.15) &\text{if growth rate} < 0.
  \end{cases}
\]

Note that since this is a one-to-one transformation ...

In this dataset there were two measures of the size of a municipality: population and population density. Unsurprisingly, the correlation between these two measures was relatively high ($0.67$). In the interest of parsimony and to mitigate possible colinearity issues, we decided to consider only one of these variables to construct our model. Population density had a moderate correlation with mean income per person ($0.49$) whereas population had a relatively low correlation with income ($0.29$). To hedge against problems with colinearity, we decided to consider population for the model building process.


\end{document}
